\begin{document}
\renewcommand{\labelenumii}{\arabic{enumi}.\arabic{enumii}}
\renewcommand{\labelenumiii}{\arabic{enumi}.\arabic{enumii}.\arabic{enumiii}}
\renewcommand{\labelenumiv}{\arabic{enumi}.\arabic{enumii}.\arabic{enumiii}.\arabic{enumiv}}

\section{Schriftliche Arbeitsaufteilung}
\setauthor{Lukas Starka}

\begin{enumerate}
    \item Einleitung (Felix Dumfarth)
    \begin{enumerate}
        \item Struktur der Diplomarbeit
        \item Ausgangslage
        \item Istzustand
        \item Problemstellung
        \item Aufgabenstellung
        \item Use-Cases
    \end{enumerate}
    \item Technischer Hintergrund (Lukas Starka)
    \begin{enumerate}
        \item Text Analysis
        \item Natural Language Processing
    \end{enumerate}
    \item Toolstack (Felix Dumfarth)
    \begin{enumerate}
        \item Programmiersprachen
        \item Technologien
        \item Werkzeuge
    \end{enumerate}
    \item Chatbots am Beispiel von Rasa (Lukas Starka)
    \begin{enumerate}
        \item Allgemeines
        \item Pipeline
        \item Welche Rolle spielen neuronale Netze in Rasa
        \item Komponenten
        \item Initialisieren
        \item Trainieren
        \item Interagieren
    \end{enumerate}
    \item Implementierung (Felix Dumfarth)
    \begin{enumerate}
        \item Systemarchitektur
        \item Backend
        \item Chat Widget
        \item Dashboard
        \item Einbindung in WordPress
        \item Deployment
    \end{enumerate}
    \item Evaluation (Felix Dumfarth)
    \begin{enumerate}
        \item Schwierigkeiten bei der Umsetzung
        \item Fazit
    \end{enumerate}
\end{enumerate}
\end{document}
