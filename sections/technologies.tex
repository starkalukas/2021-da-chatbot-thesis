\section{Programmiersprachen}
\setauthor{Felix Dumfarth}

\subsection{Java}

\begin{figure}[hbt!]
    \centering
    \includegraphics[scale=0.5]{pics/java}
    \caption{Java Logo\cite{java}}
    \label{fig:impl:java}
\end{figure}

Für das Backend haben wir uns für Java entschieden. Java ist laut dem TIOBE-Index\cite{tiobe} eine der populärsten Programmiersprachen.
Java ist eine objektorientierte Programmiersprache und besitzt dadurch Klassen und Vererbung\cite{java}.

\subsection{Python}
Phyton ist eine Programmiersprache die mehrere Arten der Programmierung unterstützt wie zum Beispiel die objektorientierte, die aspektorientierte und die funktionale Programmierung.
Phyton wird oft für Machine Learning verwendet.

\subsection{Typescript}
Typescript ist eine Programmiersprache die eine kompakte und einfache Syntax zur Programmierung von Webseiten und Anwendungen bietet.
Typescript baut auf Javascript auf und hilft zum Beispiel beim frühzeitigen Erkennen von Fehlern.

\section{Technologien}
\setauthor{Felix Dumfarth}

\subsection{REST Service}
REST steht für Representational State Transfer.
REST Anfragen sind CRUD – Operationen (Create, Read, Update, Delete).
Diese sind GET, POST, PUT und DELETE Requests.
Außerdem gibt es noch OPTIONS, PATCH, HEAD, TRACE
und CONNECT diese werden aber in der vorliegenden Arbeit nicht benutzt.

* Die GET Methode ist zum Abfragen da, es soll ein Request geschickt werden und nur Daten zurückgeben werden, es soll jedoch auf dem Server wohin die Anfrage geschickt worden ist nichts geschehen außer Daten lesen.

* Bei POST sollen Daten hinzugefügt werden, im Standardfall ist in der Respone der URI der neu gespeicherten Daten

* Bei PUT sollen Daten hinzugefügt werden, falls diese schon existieren werden Sie upgedatet und falls nicht wird ein neuer Eintrag erstellt.

* Bei DELETE sollen Daten gelöscht werden.

\subsection{Quarkus}\label{quarkus}
Quarkus ist ein Framework für Java welches sich auf die REST-Service-API ausrichtet.

\subsection{Rasa}


\subsection{Angular}
Angular ist ein auf TypeScript basierendes Framework für Webseiten.
Gedacht ist es für Single Page Applications.



\section{Werkzeuge}
\setauthor{Felix Dumfarth}
\subsection{IntelliJ IDEA}
IntelliJ ist ein IDE welche eine Vielzahl von Programmiersprachen unterstützt und für diese Arbeit verwendet wird.

\subsection{GitHub}
GitHub ist ein Online-Repository welches die gemeinsame Programmierung von Programmierern und Entwicklern erleichtert.

\subsection{Docker}
Docker ist einer der bekanntesten Container-Manager.
