\begin{spacing}{1}
    \chapter*{Abstract}
\end{spacing}
\begin{wrapfigure}{r}{0.3\textwidth}
    \begin{center}
      \includegraphics[width=0.2\textwidth]{pics/LeonieTrans.png}
    \end{center}
\end{wrapfigure}
This diploma thesis deals with the topic of a chatbot that was created with the help of Rasa and can be found on the HTL Leonding website.
This chatbot should be able to provide answers about school-specific topics quickly and easily.
The bot's topics range from school-specific questions to small talk and general information about the work.
The users have the opportunity to rate the given answers to their questions and to give feedback.
In addition, a so-called dashboard was developed, in which an overview of all the conversations with the chatbot can be seen.
With a built-in editor, the knowledge of the bot can be constantly expanded.
\newpage
\begin{spacing}{1}
    \chapter*{Zusammenfassung}
\end{spacing}
\begin{wrapfigure}{r}{0.3\textwidth}
    \begin{center}
      \includegraphics[width=0.2\textwidth]{pics/LeonieTrans.png}
    \end{center}
\end{wrapfigure}
Die vorliegende Diplomarbeit beschäftigt sich mit dem Thema eines Chatbots, der mithilfe von Rasa erstellt worden ist und auf der Website der HTL Leonding zu finden sein soll.
Dieser Chatbot soll interessierten Personen einfach und schnell Antworten über schulspezifische Themen liefern können.
Dabei reichen die Themengebiete des Bots von schulspezifischen Fragen bis hin zu Small Talk und generellen Informationen über die Arbeit.
Den Benutzerinnen und Benutzern steht dabei die Möglichkeit zu, die Beantwortung ihrer Fragen zu bewerten und Feedback zu geben.
Zusätzlich wurde noch ein sogenanntes Dashboard entwickelt, bei dem eine grafische Aufbereitung aller Unterhaltungen mit dem Chatbot zu sehen ist.
Mit einem eingebauten Editor kann dabei das Wissen des Bots ständig erweitert werden.