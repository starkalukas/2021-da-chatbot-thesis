\section{Schwierigkeiten in der Umsetzung}
\setauthor{Felix Dumfarth}

\subsection{Backend}
Beim Backend war eine große Schwierigkeit, dass die Rasa X API nicht sehr gut dokumentiert ist und deshalb sehr viele der Endpoints nur durch Ausprobieren ermittelt werden konnten.


\subsection{Frontend}
Beim Frontend war die größte Schwierigkeit der Export der Chatkomponente, da Angular Materials nicht mit-exportiert werden konnte, dies jedoch für die Feedback-Ansicht notwendig ist.

\section{Fazit}
Das geplante Ergebnis der Diplomarbeit war laut dem Diplomarbeitsantrag:

''Ein webbasierter grafischer Avatar, der durch eine modulare Architektur leicht in Webseiten integrierbar ist.
Wichtig dabei ist, dass dieser selbstlernend ist und nicht rule-based.
Das Wartungspersonal soll dabei Logs des Chatbots erhalten und die Möglichkeit haben, die Wissensbasis zu manipulieren, also neue Inhalte bereitzustellen oder nicht mehr relevante Inhalte zu entfernen.''

In der Arbeit kann man erkennen, dass ein Großteil dieser Punkte erfüllt worden ist.

\begin{itemize}
    \item Da das Chat-Widget als Webkomponente exportiert wurde, ist es sehr leicht, den Chat in Webseiten zu integrieren.
    \item Selbstlernend ist er nicht.
    Der Chatbot verwendet zwar Machine Learning, um Muster zu erkennen, damit der Text auch erkannt wird, wenn er nicht wie in den Trainingsdaten eingegeben wurde.
    Dabei lernt der Bot allerdings nicht von alleine neue Intents.
    \item Im Dashboard erhält das Wartungspersonal alle Unterhaltungen.
    \item Im Dashboard kann das Wartungspersonal die Wissensbasis manipulieren.
\end{itemize}

Angefangen hat diese Arbeit mit Recherchen über die Geschichte der Chatbots, von Eliza zum Google Assistant, sodass herausgefunden werden kann, wie die Chatbots in der Vergangenheit gearbeitet haben.
Es wurde unter anderem mit Rasa, Dialogflow und Keras herumexperimentiert und kleine Prototypen angefertigt, um herauszufinden, welche Technologie benutzt werden sollte.
Es hat eine lange Zeit gedauert, bis der richtige Weg für diese Arbeit gefunden wurde.
Doch als dieser gefunden wurde, hat sich unser Chatbot rasend schnell entwickelt.
Beim Frontend wurden Prototypen angefertigt und mit Farben und Design experimentiert, bis das passende gefunden wurde.
Das Endprodukt dieser Arbeit kann noch sehr viel erweitert werden.
Es könnten Rollen beim Dashboard eingeführt werden oder eine englische Version des Chatbots für eine englische Version der Schulhomepage.
Das Erweiterungspotenzial ist sehr groß, nichtsdestotrotz ist der Chatbot und das Dashboard jetzt auch schon ein gut funktionierendes Produkt.
Wenn der Chatbot auf der Schulhomepage sein permanentes Zuhause findet, wird er sicher vielen interessierten Personen mit seinem Wissen unterstützen können.
