Das geplante Ergebnis der Diplomarbeit war laut dem Diplomarbeitsantrag:

Ein webbasierter grafischer Avatar, der durch eine modulare Architektur leicht in Webseiten integrierbar ist.
Wichtig dabei ist, dass dieser selbstlernend ist und nicht rule-based.
Das Wartungspersonal soll dabei Logs des Chatbots erhalten und die Möglichkeit haben, die Wissensbasis zu manipulieren, also neue Inhalte bereitzustellen oder nicht mehr relevante Inhalte zu entfernen.

In der Arbeit kann man erkennen das ein großteil dieser Punkte erfüllt worden sind.

\begin{itemize}
    \item Da das Chat-Widget als Webkomponente exportiert wurde, ist es sehr leicht den Chat in Webseiten zu integrieren.
    \item Selbstlernend ist er nicht da dies in dieser Form nicht möglich ist und praktisch gewesen wäre, aber er durch ML erkennt er Muster.
    \item Im Dashboard erhält das Wartungspersonal alle Unterhaltungen.
    \item Im Dashboard kann das Wartungspersonal die Wissensbasis manipulieren.
\end{itemize}

\section{Schwierigkeiten in der Umsetzung}
\setauthor{Felix Dumfarth}

\subsection{Backend}
Beim Backend war eine sehr große Schwierigkeit das die Rasa X API nicht sehr gut dokumentiert ist und deshalb sehr viele der Endpoints nur durch Probieren und sehr tiefe Recherchen ermittelt werden konnten.


\subsection{Frontend}
Beim Frontend war die größte Schwierigkeit der export der Chatkomponente da Angular Materials nicht mit exportiert werden konnte, dies jedoch für die Feedback-Ansicht notwendig ist.
%TODO Lösung wenn i eine find
